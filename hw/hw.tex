\documentclass[lang=cn,11pt]{elegantbook}
\usepackage[utf8]{inputenc}
\usepackage[UTF8]{ctex}
\usepackage{amsmath}%
\usepackage{amssymb}%
\usepackage{graphicx}

\title{597 Homeworks}

\begin{document}
\frontmatter
\tableofcontents
\mainmatter



\chapter{Homework 1}

\section{Problems from Chapter 1 of Ross}
Attempt problems from Chapter 1 of Ross as follows:

(i) From the Problems section of the textbook: 21, 22, 34.

(ii) From the Theoretical Exercises section of the textbook: 20, 21, 22.

\section{Problems from Chapter 2 of Ross}
Solve the following problems from Chapter 2 of Ross:

(i) From the Problems section of the textbook: 45, 46.

(ii) From the Theoretical Exercises section of the textbook: 15, 19, 20.

\section{Multinomial Theorem}
Prove the multinomial theorem:
\[
(x_1 + x_2 + \cdots + x_r)^n = \sum_{0 \leq n_1, \ldots, n_r \leq n, \; n_1+n_2+\cdots+n_r=n} \frac{n!}{n_1! n_2! \cdots n_r!} x_1^{n_1} x_2^{n_2} \cdots x_r^{n_r}.
\]

\section{Integral Representation of Factorial}
Show that:
\[
n! = \int_0^\infty t^n e^{-t} \, dt.
\]
This is a useful formula to remember.

\section{Generalized Leibniz Rule for Derivatives}
(i) Let $f$ and $g$ be two infinitely differentiable functions. Find the formula for:
\[
\frac{d^n}{dx^n}(fg)
\]
as a sum of products of derivatives of $f$ and $g$, generalizing the Leibniz rule \((fg)' = f'g + fg'\). Hint: Compare to the binomial formula to make an educated guess, then prove it by induction.

(ii) Extend the result to the product of $r$ functions:
\[
\frac{d^n}{dx^n}(f_1 \cdots f_r).
\]

\section{Path Counting in a Grid}
Consider a grid of size $n \times n$. Starting from the lower-left corner, move to the upper-right corner, only going right or up at each step.

(i) Determine how many such paths are possible.

(ii) Let $a_n$ denote the number of paths that always remain below the diagonal (i.e., if $(i, j)$ is a point on the path, then $j \leq i$). Prove that:
\[
a_n = \sum_{k=1}^n a_{k-1} a_{n-k}.
\]

(iii) Use induction to show that:
\[
a_n = \frac{1}{n+1}\binom{2n}{n+1}.
\]













































\end{document}